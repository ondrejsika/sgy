% Sika Gymansium Paper
% http://sgy.cz/sgy.pdf
% Author: Ondrej Sika <ondrej@ondrejsika.com>

\documentclass[12pt,a4paper]{article}
\usepackage[utf8]{inputenc}
\usepackage[margin=2cm]{geometry}
\usepackage[parfill]{parskip}
\pagestyle{empty}

\begin{document}

% HEADER

\begin{center}
{\LARGE \bf Sika Gymnasium}\\
\vspace*{0.4cm}
{\large Ondrej Sika {\tt <ondrej@ondrejsika.com>}}\\
\vspace*{0.3cm}
2015\\
\end{center}

% BODY

\section*{Proc jsem se rozhodl zalozit gymnasium}

System skol, ktery tu mame dnes je z roku 1850 a neni prilis vhodny pro dnesni dobu. Proto se pokousim o zmenu, ktera musi byt zasadni.


\section*{Soukroma nebo statni skola}

Jsem zasadne proto, aby skola byla soukroma. Ale zaroven neziskova. I prez nazory ve spolecnosti ze soukrome skoly jsou spatne, chci aby skola mohla dynamicky reagovat na soucasne deni a to u statni skoly je velky problem.


\section*{Cil gymnazia}

Chci aby v mem gymnasiu studenti neziskavali informace. Ve tride je ucite a zaci. Ucitele uci zakum svoji pravdu, predava nejake informace. Zaci si tyto informace pamatuji. Tento model vznikl v roce 1850, kdy byl nedostatek informaci a prostrednictim skoly se predavali dal. Dnes je ale uplne jina doba, diky internetu je informaci dostatek, ne-li prebytek a cilem stredni skoly by nemelo byt informace predavat informace, ale naucit studenty s nimi pracovat.


\section*{Financovani}

Chci zalozit soukromou skolu. Financovani chci hradit ze soukromich zdroju, predevsim z daru a skolneho. Pokud se rozpocet skoly vejde do techto druhu prijmu, bude to idealni. Pokud by mel problemy, je moznost ziskat nejake dotace, ovsem je to v rozporu s mymi idealy. Jsem nazoru ze dotace jsou spatne, presto si myslim, ze v tomto pripade by byly penize vyuzity na spravnou vec.


\section*{Vyuka}

Forma vyuky bude odlisna od te soucasne. Jak jsem jiz psal vyse, je potreba ji zmenit. Chtel bych aby zaci nebyli tupymi posluchai informaci ale aby si informace sami vyhledavali a sami se rozhodovali, ktere pro ne maji nejaky vyznam. U nekde se tato metodika da pouzit dobre, nekde hur, proto to neni dogma. Nekdy je pro zaky jednodussi jim pomoci s hlednanim a ukazat jim smer, ale zaroven je nenutit po nem jit.

Chci aby skola mela svoje studijni materialy a svoji metodiku, ale aby nenutila zaky se ji striktne drzet.

Znamkovani neni uplne nejstasnejsi system protoze pokud nechcete vsechny studenty vest po stejne ceste, jejich znalosti jsou neporovnatelne. Proto si myslim, ze znamkovani neni potreba, pokud maji studenti o vzdelani zajem, v opacenem pripade ani znamkovani nepomuze. Vysledkem rocni nebo pulrocni navstevy skoly by nemela byt zkouska typu otazka odpoved, ale nejaka vlastni prace, ktera ma smysl.

Vyuka vetsiny predmetu by mela probihat v anglictine a vsechny materialy by take mely byt anglicky. Jsem si jist ze pouzivani anglictiny zlepsi jeji znalost studentum a zaroven anglicka vyuka otevira dvere pro zahranicni studenty. Nektere predmety jako Cesky jazyk by se vyucovali stale v cestine.

\section*{Ucitele}

Chtel bych zajistit aby ucitele meli dobre znalosti a praxi v oboru ktery budou ucit. Chci aby mohli zakum predat to nejlepsi a aby mohli moderovat diskuze zaku a dat jim pridanou hodnotu.

Tyto lide jsou velmi drazi, proto bych urcite nebudou mit plny uvazek. Na kazde tema muze byt jiny ucitel, podle me to nicemu nevadi. Pouze tridni ucitele budou porad stejni, tam si myslim, ze je to dobre.


\section*{Akreditace}

Akreditace skoly je dulezita, ale pokud ma byt v rozporu se zakladnim principem skoly a nebo na ukor vzdelani zaku, jsem si jist ze dobra skola muze fungovat i bez statni akreditace.

University by meli pri vyberu studentu dbat na jejich znalosti a ne na jejich formalni vzdelani. Stejne tak jako to funguje vsude v soukrome sfere.


\end{document}

